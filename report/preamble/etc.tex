%!TEX root = ../Thesis.tex

\usepackage{mathtools}    % Det meste matematik (indeholder ams­math og rettelser)
\usepackage{amssymb}
\usepackage{bm}           % Bold symbols
\usepackage{xfrac}        % Flere fracs (\sfrac{}{})
\usepackage{listings}     % Indsæt code
\usepackage{todonotes}    % Cool to-do notes, [disable] skjuler to-dos
\usepackage[backend=biber,bibstyle=ieee,citestyle=numeric-comp]{biblatex} % Benyt BibLaTeX til formatering
\usepackage{subcaption}   % Tillader subfigure, subtable samt \captions
\usepackage{csquotes}
\usepackage{afterpage}
\usepackage{placeins}
\usepackage{algorithm}
\usepackage[noend]{algpseudocode}


% algorithm environment
%http://tex.stackexchange.com/questions/1375/what-is-a-good-package-for-displaying-algorithms
\algnewcommand{\Let}[2]{\State #1 $\gets$ #2}
\algnewcommand{\Not}[0]{\textbf{not }}
\newcommand{\Implicit}[1]{\State \textit{#1}}
\algnewcommand{\LineComment}[1]{\State \(\triangleright\) #1}
\algrenewcommand\Call[2]{\textproc{#1}(#2)}
\algrenewcommand\alglinenumber[1]{{\footnotesize\color{gray}\ttfamily#1}}

%listing settings, æøå support, font config, line number, left lines
\lstset{
    breakatwhitespace=false, breaklines=true,
    inputencoding=utf8, extendedchars=true,
    literate={å}{{\aa}}1 {æ}{{\ae}}1 {ø}{{\o}}1 {Å}{{\AA}}1 {Æ}{{\AE}}1 {Ø}{{\O}}1,
    keepspaces=true, showstringspaces=false, basicstyle=\small\ttfamily,
    frame=L, numbers=left, numberstyle=\scriptsize\color{gray},
    keywordstyle=\color{SteelBlue}\ttfamily,
    stringstyle=\color{IndianRed}\ttfamily,
    commentstyle=\color{Teal}\ttfamily,
} 

\DeclareGraphicsExtensions{.pdf,.eps,.png,.jpg,.gif}	% ændre til .png, .jpg for hurtig visning

\newcommand\defeq{\mathrel{\overset{\makebox[0pt]{\mbox{\tiny def}}}{=}}}
