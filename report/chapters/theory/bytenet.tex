\section{ByteNet}

The ByteNet model is an alternative approach to neural machine translation, it specifically focuses on having linear computational complexity, having a resolution preserving encoding, and be a charecter-level translation model \cite{bytenet}. This is somewhat diffrent from existing popular models such as the Sutskever 2014 model \cite{sutskever-2014-nmt} and the attention based Bahdanau 2015 model \cite{bahdanau-2015-nmt}.

Note that the model presented here is a simplified version of the original ByteNet model \cite{bytenet}. First, there are no bags of characters, only individual characters are used as the sentence representation. Secondly, the Sub-Batch normalization in the decoder is replaced by Layer Normalization, besides being a simpler solution than Sub-Batch Normalization this could also improve the model performance \cite{layer-normalization}.

\subsection{ByteNet Residual Block}

Before explaining the full ByteNet network, a central component of the ByteNet model called the \textit{ByteNet residual block} needs to be explained first. There are two variations of this an \textit{Encoder Residual Block}, and a \textit{Decoder Residual Block}. First the encoder version is explained and then some slight alterations of this will turn it into the decoder version. For a visual representation see Figure \ref{fig:bytenet:residual-block}.

\begin{figure}[h]
    \centering
    \begin{subfigure}[b]{0.45\textwidth}
        \centering
        \includegraphics[scale=1]{theory/bytenet-residual-block.pdf}
        \caption{Residual Block used in encoder.}
    \end{subfigure}
    ~ %
    \begin{subfigure}[b]{0.45\textwidth}
        \centering
        \includegraphics[scale=1]{theory/bytenet-residual-block-causal.pdf}
        \caption{Residual Block used in decoder.}
    \end{subfigure}
    \caption{The residual blocks used in the ByteNet model. The blocks are denoted by Encoder-Residual-Block$(k,r)$ and Decoder-Residual-Block$(k,r)$, where $k$ is the kernel width and $r$ is the dialation rate.}
    \label{fig:bytenet:residual-block}
\end{figure}

\subsubsection{Encoder Residual Block}

The \textit{ByteNet residual block} is at its core a dialated convolution, but adds upon this many of the ideas presented in section \ref{sec:convergence}. The entire block is a \textit{residual layer}, meaing that that the input is added to the final output of the transformation function $\mathcal{F}$. The transformation can be logically grouped into a set of sub-blocks, a normalization block, a dimentional reduction block, a dialated convolution block, and finally a dimentional restoration block.

The normalization block consists of Batch-Normalization and a ReLU activation. For this to make sense there must be some parameters before the activation such that the network can control the ReLU. This will either be an embedding lookup or another \textit{residual block}.

The dimentional reduction block reduces the dimentionality of the dataset from $d$ dimentions to $\frac{1}{2}d$ dimentions. This is done though the usual Dense Layer, Batch-Normalization, and ReLU acktivation block sequence. Note that the input is a sequence of vectors, a vector for character, thus the dense matrix multiplication is performed on each vector in the sequence. This can also be descibed as a normal convolution with a kernel width of one.

\afterpage{\clearpage}

The dialated convolution block depends on two parameters (or 3 if dimention size $d$ is included), the kernel width $k$ and the dilation rate $r$. ByteNet specifies a specific structure for these parameters, but these choices only makes sense from the full network perspective, thus they will be discussed later. Besides being a dialated convolution the sub-block is quite normal, as it also performs a Batch-Normalization, and ReLU acktivation. The convolution also maintains the dimentionality, which at this point is $\frac{1}{2}d$.

For the \textit{residual layer} to make sense, the output dimention must be equal to the input dimention, thus the final sub-block is just a dense layer that transforms from $\frac{1}{2}d$ dimentions to $d$ dimentions. This sub-block does not contain either a Batch-Normalization, or a ReLU activation. Though these are implicitly performed if another \textit{residual block} follows.

\subsubsection{Decoder Residual Block}

The decoder is very similar to the encoder, the differences exists such that the decoder can't look into the future of the target sequence. If the decoder was allowed to look into the future it would be impossible to perform inference, when the target sequence is unknown. To prevent future inspection two changes are made‚ the dialiated convolution is masked, and Batch-Normalization is replaced with Layer-Normalization.

Masking the convolution means that only the part of the convolution kernel that looks at the past is preserved. This is equivalent to fixing the right side (the future side) of the preserved kernel to zero.

\subsection{Hierarchical Dilated Convolution}
\label{sec:theory:bytenet:hierarchical-dilated-convolution}

Each \textit{ByteNet Residual Block} can conceptually be seen as just a modified dilated convolution. This conceptual model is usefull for understanding the main trick that makes \textit{ByteNet} so powerful, \textit{Hierarchical Dilated Convolution}.

The idea behind \textit{Hierarchical Dilated Convolution} is that by expoentially increasing the dilation rate in stacked dilated convolution layers, the effective kernel width will expoentially increase for each layer, while the amount of weight in the kernel only increases linearly. The encoder part (bottom half) of figure \ref{fig:bytenet:simplified-hdc} helps to understand this idea in detail.

\begin{figure}[h]
    \centering
    \includegraphics[scale=1]{theory/bytenet-hierarchical-dialted-convolution-enc-dec.pdf}
    \caption{A simplified version of the ByteNet model, showing the interaction between the decoder and encoder.}
    \label{fig:bytenet:simplified-hdc}
\end{figure}

Let the dilation rate for the encoding layers $\ell \in [1, L_{enc}]$ be denoted $r_\ell$, the exponential increasing dilation rate used in the ByteNet model is then $r_\ell = 2^{\ell - 1}$. The top layer $(L_{enc})$ will thus have a width of $(k-1) 2^{L_{enc} -1} + 1$. The effective total width is slightly larger, because the layer below have a width of $(k-1) 2^{L_{enc} - 2} + 1$. This pattern continues down to the first layer, resulting in a effective total width of:
\begin{equation}
\sum_{\ell = 1}^{L_{enc}} (k - 1) 2^{\ell-1} + 1 = (k - 1) (2^{L_{enc}} - 1) + 1
\end{equation}
The kernel size however is $k \times d$ for all layers, thus the total amount of weights is:
\begin{equation}
\sum_{\ell = 1}^{L_{enc}} k d = L_{enc} k d
\end{equation}

In the original ByteNet model they uses 4 dilated convolution layers, each with a kernel width of $5$, this results in an effective kernel width of 125 charecters (actaully it is 373 charecter wide, because they repeat the pattern 3 times). Having a wide but computationally cheap kernel is extreamly important for a character-based translation model such as ByteNet. In an attention-based translation model, such as the Bahdanau model \cite{bahdanau-nmt} the width is in theory arbitrarily wide (though in practice very wide attention mechanisms works poorly), however as discussed previously this approach causes a qudradic computational complexity (section \ref{sec:theory:sequential:bahdanau}). By \textit{hierarchical dilated convolution} ByteNet maneges to keep the complexity linear, while still haveing a very wide kernel. In practise sentences have a limited length and thus a hierarchical dilated convolution approach can be compareable to an attention approch. In the WMT 2014 en-de dataset, the longest sentences are 400 characters wide (Figure \ref{fig:bytenet:wmt-deen-density}).

\begin{figure}[h]
    \centering
    \includegraphics[scale=1]{theory/wmt-deen-density.pdf}
    \caption{Density estimate of the sentence length in the WMT 2014 en-de dataset.}
    \label{fig:bytenet:wmt-deen-density}
\end{figure}

\subsection{The Network}

The full \textit{ByteNet} model combines the \textit{Residual Blocks}s with the \textit{Hierarchical Dilated Convolution} as dicussed earlier. The \textit{Hierarchical Dilated Convolution} pattern is then repeated 3 times for both the encoder and decoder, as vislalized in figure \ref{fig:bytenet:network}.

The input to the encoder is the source sequences mapped using an embedding matrix. The input to the decoder is the encoded sequence concatenated with the previuse target prediction. This target prediction is then also mapped using a diffrent embedding matrix. Note in particular that because the encoded vector is concatenated with the target sequence vector, the internal dimentionality of the decoder is twice that of the encoder. The decoded output is finally projected using a dense layer such the output has vocabulary sized dimensionality. After this a softmax is used to transform the output into the properbility domain.

An interresting effect of this network is that all layers are invariant to adding bais term. The encoder embedding and encoder layers are invariant because of Batch-Normalization. The decoder embedding and decoder layers are invariant because of Layer-Normalization. The final dense layer is invariant because of the following softmax.

\begin{figure}[h]
    \centering
    \includegraphics[scale=1]{theory/bytenet-network.pdf}
    \caption{The training network for the ByteNet Model, where the shifted target sequence is feed into the network. In the inference network the shifted target sequence is replaced by the $\mathrm{argmax}(\cdot)$ of the softmax output. ResBlock$(k, r)$ in the encoder refers to the Encoder-Residual-Block$(k, r)$, in the decoder it refers to Decoder-Residual-Block$(k, r)$.}
    \label{fig:bytenet:network}
\end{figure}

\subsubsection{Training versus Inference}
The ByteNet model actually uses a different network for training than it does for inference. This is not strictly necessary, as one could use the same network for both training and interference like in a FFNN, however for sequental models the optimization can often converge faster if the target sequence is feed into the model.

Feeding the target sequence to the network is an optimization trick that may seam counter intutive at first. The idea is that because prediction of the current charecter depends on the prediction of the previuse charecters ($\mathbf{y}_t = f(\mathbf{x}_{1:|S|}, \mathbf{y}_{1:t-1})$), then instead of letting the prediction cascade, which will also cause prediction errors to cascade, $\mathbf{y}_{1:t-1}$ is replaced with the known target sequence such that $\mathbf{y}_t = f(\mathbf{x}_{1:|S|}, \mathbf{t}_{1:t-1})$.

This trick also has the added benefit that training can be parallelized over target sentence and not just the observations and source sentence. 

\todo[inline]{There are some indications of this being suboptimal \url{https://arxiv.org/abs/1506.03099}}
